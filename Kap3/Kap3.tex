\chapter{Caminatas cuánticas sobre la línea}

\section{Idea original}\label{IdeaOriginal}
La idea seminal de \cite{aharonov2001quantum} se basa en que hacer mediciones proyectivas a un estado en superposición es un evento aleatorio. Los pasos del caminante dependen de dicha aleatoriedad.\\

Consideremos una partícula con espín $1/2$ capaz de trasladarse en 1 dimensión (a lo largo de una línea). Su posición en $t$ se describe por un paquete de onda $\psi_{x_0}(t)$ centrado en $x_0$ y su espín $s$ se describe en la base de $\hat{S}_z$, $\{\ket{\downarrow},\ket{\uparrow}\}$. 
El espacio completo de la caminata es $\mathcal{H}=\mathcal{H}_S\otimes\mathcal{H}_P=\text{span}\left\{\ket{\sigma}\otimes \ket{\psi_{x_0}}|\sigma=\{\downarrow,\uparrow\},\,x_0\in \mathbb{Z}\right\}$. La amplitud $\ket{\Psi}$ es

\begin{equation}
    \ket{\Psi}=\alpha^{\uparrow}\ket{\uparrow}\otimes\ket{\psi^\uparrow}+\alpha^\downarrow\ket{\downarrow}\otimes\ket{\psi^\downarrow}
\end{equation}{}

en donde $\braket{\psi^\uparrow|\psi^\uparrow}=\braket{\psi^\downarrow|\psi^\downarrow}$, y $|\alpha^\uparrow|^2+|\alpha^\downarrow|^2=1$.\\

Como es usual, las traslaciones espaciales $\hat{T}$ son generadas por el momento $\hat{p}$ de la partícula, $\hat{T}=e^{-i\hat{p}l}$:

\begin{equation}
    \hat{T}_{\pm l}\ket{\psi_{x_0}}=\ket{\psi_{x_0 \pm l}}\qquad
    \begin{array}{cc}
    \hat{S}_z\ket{\uparrow}=+\frac{\hbar}{2}\ket{\uparrow}\\
    \hat{S}_z\ket{\downarrow}=-\frac{\hbar}{2}\ket{\downarrow}
    \end{array}
\end{equation}

Como hemos anticipado, la traslación de la partícula depende del espín: en la base de $S_z$, el espín arriba genera una traslación hacia la derecha y el espín abajo una traslación hacia la izquierda. Así que la evolución es generada por $\hat{U}=e^{-2i\hat{S}_z\otimes \hat{p}l}$. Sea un estado inicial $\ket{\Psi_{\text{in}}}=(\alpha^{\uparrow}\ket{\uparrow}+\alpha^{\downarrow}\ket{\downarrow})\otimes\ket{\psi_{x_0}}$,

\begin{equation}
\hat{U}\ket{\Psi_{in}}=\alpha^\uparrow \ket{\uparrow}\otimes\ket{\psi_{x_0+l}}+\alpha^\downarrow\ket{\downarrow}\otimes\ket{\psi_{x_0-l}}
\end{equation}{}

Al medir sobre el espacio de espín obtenemos el estado $\ket{\psi_{x_{0+l}}}$ con probabilidad $p^\uparrow=|\alpha^\uparrow|^2$, o el estado $\ket{\psi_{x_{0-l}}}$ con probabilidad $p^\downarrow=|\alpha^\downarrow|^2$, correspondientes a un desplazamiento $\pm l$. Después de esta medición el estado de espín queda definido arriba o abajo, por lo cual una nueva traslación tiene también un desplazamiento definido, en otras palabras, desaparece la aleatoriedad. 
Para dar el siguiente paso en forma aleatoria es necesario reponer el estado de espín superpuesto, de otra manera, el movimiento tiene la dirección única del estado de espín proyectado con la medición. Es necesario que los pasos de la caminata sean la sucesión de una traslación y la recuperación del estado de espín inicial tras cada paso. Como sea, podemos retomar el factor aleatorio del proceso si hacemos mediciones sobre distintas bases cada vez. Por ejemplo, tras haber proyectado sobre el eje $z$, una rotación de la base de medida en el ángulo $\theta$, presenta una superposición $\ket{\uparrow}=c_1\ket{\xrightarrow{}}+c_2\ket{\xleftarrow{}}$. Es equivalente una rotación del espín, a partir, por ejemplo, de la matriz de rotaciones típica $R(\theta)$\footnote{\begin{equation}
    R(\theta)=
    \begin{pmatrix}
    \cos\theta&-\sin\theta\\
    \sin\theta&\cos\theta
    \end{pmatrix}{},
    \label{MatrizRotaciones}
\end{equation}{}},

Comprobemos el resultado de $\hat{M}_zR(\theta)\hat{U}\ket{\Psi_{in}}$ donde $\hat{M_z}$ es una medición en la base de $\hat{S}_z$.
\begin{equation}
    \hat{U}=e^{-2iS_z\otimes Pl}=e^{-i(\ket{\uparrow}\bra{\uparrow}-\ket{\downarrow}\bra{\downarrow})\otimes \hat{P}l}
\end{equation}{}
Los dos términos de la exponencial conmutan, por lo cual es igual a $e^{-i\ket{\downarrow}\bra{\downarrow}\otimes \hat{P}l}e^{\ket{\uparrow}\bra{\uparrow}\otimes \hat{P}l}$. Cada exponencial se puede simplificar:

\begin{equation}
e^{\ket{\uparrow}\bra{\uparrow}\otimes\hat{P}l}=\sum_n \frac{(\ket{\uparrow}\bra{\uparrow})^n\otimes (\hat{P}l)^n}{n!}=\ket{\uparrow}\bra{\uparrow}\otimes\sum_n\frac{(\hat{P}l)^n}{n!}=\ket{\uparrow}\bra{\uparrow}\otimes e^{\hat{P}l},
\end{equation}
porque $\ket{\uparrow}\bra{\uparrow}$ es un proyector.\\

Si la longitud del paso es mucho menor que el tamaño del ancho de la función de onda, $x_0\gg l$, podemos tomar hasta primer orden en la exponencial como buena aproximación y operar:
después de que la operación está completa, es decir, la proyección sobre $\hat{S_z}$ seguido de la traslación $\hat{T}$ ya se completaron, la medición $M_z$ 

\begin{equation}
    M_zR(\theta)U\ket{\Psi_{in}}=
    \left\{
    \begin{array}{cc}
    \ket{\uparrow}\otimes(I-iPl\delta^\uparrow)\ket{\psi_{x_0}}\\
    \ket{\downarrow}\otimes(I-iPl\delta^\downarrow)\ket{\psi_{x_0}}
    \end{array}{}
    \right.
\end{equation}{}
con probabilidades, $p^\uparrow=|\alpha^\uparrow\cos{\theta}-\alpha^\downarrow\sin\theta|^2$, y $p^\downarrow=|\alpha^\uparrow\sin{\theta}+\alpha^\downarrow\cos\theta|^2$, de obtener el respectivo estado de posición.

\begin{equation}
    l\delta^\uparrow:=l\dfrac{\alpha^\uparrow \cos{\theta+\alpha^\downarrow\sin{\theta}}}{\alpha^\uparrow\cos\theta-\alpha^\downarrow\sin\theta}\qquad l\delta^\downarrow:=l\dfrac{\alpha^\uparrow \sin{\theta-\alpha^\downarrow\cos{\theta}}}{\alpha^\uparrow\sin\theta-\alpha^\downarrow\cos\theta}
\end{equation}{}

Un único paso tiene la probabilidad de ser muy grande y llevar a la partícula muy lejos, o ser mucho menor que $l$, y prácticamente no generar traslación. Como ejemplo extremo tomemos $\tan\theta=|\alpha^{\uparrow}/\alpha^{\downarrow}|(1+\epsilon)$, con $l/ \Delta x \ll |\epsilon| \ll1$. Si medimos $\ket{\uparrow}$ el desplazamiento resultante es $l\delta^{\uparrow}\approx -2l/\epsilon$.
Está claro que la caminata puede continuarse aplicando el producto $\hat{R}(\theta)\hat{U}$ $t$ número de veces, $\hat{R}^t\hat{U}^t$, sobre el estado inicial de la caminata.\\

La medición aproxima la caminata cuántica a su contraparte clásica. Vimos cómo el caso límite de medir tras cada evolución tiene resultados iguales al clásico. La medición anula las coherencias cuánticas.
Sin embargo, notemos que este nuevo proceso da como resultado una caminata aleatoria clásica, en la cual cada paso tiene probabilidad fija de ir hacia alguno de los dos lados.

\subsection{Un modelo con moneda}\label{sec:LineaInfinita}
En este modelo simplificamos las caminatas cuánticas en dos aspectos\cite{kempe2003quantum} : (1) el espacio de posiciones es discreto, y (2) las traslaciones son de magnitud constante e igual a $1$. (1) impone pensar en una partícula localizada puntualmente y no como un paquete de onda con dimensión finita, por ejemplo, comúnmente las trayectorias tienen posición inicial en $x=0$. 
Por otro lado, (2) restringe la posibilidad de traslaciones muy largas o muy cortas como encontramos en la sección anterior $\ref{IdeaOriginal}$, en cambio, las traslaciones posibles son de longitud definida e igual en todo paso. Con este modelo nos centramos en las consecuencias de las correlaciones cuánticas tras muchos pasos.\\

El espacio de la caminata se genera por dos subespacios, $\mathcal{H}=\mathcal{H}_C\otimes\mathcal{H}_P$: el de la moneda
$\mathcal{H}_C=\text{span}\{\ket{0},\ket{1}\}$, y el de la posición $\mathcal{H}_P=\text{span}\{\ket{x}|x\in \mathbb{Z}\}$. \\

Más allá de las simplificaciones, la evolución es equivalente a la de la sección anterior: lanzar la moneda y después dar un paso según el estado final de la moneda (rotar el espín y luego caminar de acuerdo al espín final). 
La evolución es $\hat{U}=\hat{S}(\hat{C}\otimes I_{\mathcal{P}})$. $\hat{C}$ opera sobre $\mathcal{H}_C$,  $\hat{S}$ opera sobre $\mathcal{H}_P$, y $\hat{U}$ opera sobre $\mathcal{H}$. Los tres son operadores unitarios.
Podemos escribir

\begin{equation}
\hat{S}=\ket{\downarrow}\bra{\downarrow}\otimes \sum_{x\in \mathbb{Z}}\ket{x-1}\bra{x}+\ket{\uparrow}\bra{\uparrow}\otimes \sum_{x\in \mathbb{Z}}\ket{x+1}\bra{x}
\end{equation}{}
$\hat{C}$ sólo opera sobre las direcciones.\\

Debido a que la definición de $\hat{S}$ es estricta y tiene una única finalidad, la cual es realizar las traslaciones impuestas por $\hat{C}$, es claro que las diversas caminatas que podamos configurar están contenidas en $\hat{C}$. 
Consideremos el análisis de la caminata con el famoso operador de \textit{Hadamard} $\hat{H}$, 
\begin{equation*}
\hat{H}\doteq
\begin{pmatrix}
1 & 1\\
1 & -1
\end{pmatrix}
\end{equation*}{}
que posee la cualidad de ser equilibrada: después de un paso o una aplicación, la probabilidad de la partícula de estar a la derecha o a la izquierda en ambos casos es $1/2$. Exploremos la primera caminata de este trabajo con esta moneda.\\

El siguiente es un paso desde el estado inicial $\ket{1}\ket{x=0}$
\begin{align*}
\ket{\uparrow}\otimes\ket{0}&\xrightarrow{\;\;\;\hat{H}\;\;\;}\dfrac{1}{\sqrt{2}}(\ket{\uparrow}+\ket{\downarrow})\otimes\ket{0}\\
&\xrightarrow{\;\;\;\hat{S}\;\;\;}\dfrac{1}{\sqrt{2}}(\ket{\uparrow}\otimes\ket{1}+\ket{\downarrow})\otimes\ket{-1})
\end{align*}{}
Una medición sobre el espacio de moneda en la base $\{\ket{0},\ket{1}\}$ da como resultado la posición $\ket{1}$ con probabilidad $1/2$, y $\ket{-1}$ con la misma probabilidad.
Para mantener las correlaciones nos abstenemos de medir a cada paso y lo hacemos sólo hasta el final. Por ejemplo, después de 3 pasos:
\begin{align*}
\ket{\phi_{ini}}&\xrightarrow{\;\;\;\hat{U}\;\;\;}\dfrac{1}{\sqrt{2}}(\ket{\uparrow}\otimes\ket{1}-\ket{\downarrow}\otimes\ket{-1}\\
&\xrightarrow{\;\;\;\hat{U}\;\;\;}\dfrac{1}{2}(\ket{\uparrow}\otimes\ket{2}-(\ket{\uparrow}-\ket{\downarrow})\otimes\ket{0}+\ket{\downarrow}\otimes\ket{-2})\\
&\xrightarrow{\;\;\;\hat{U}\;\;\;}\dfrac{1}{2\sqrt{2}}(\ket{\uparrow}\otimes\ket{3}+\ket{\downarrow}\otimes\ket{1})+\ket{\uparrow}\otimes\ket{-1}-2\ket{\downarrow}\otimes\ket{-1}-\ket{\downarrow}\otimes\ket{-3}),
\end{align*}{}
En el estado final es más probable obtener $\ket{x=-1}$ que cualquiera otra posición, inclusive la simétrica $\ket{+1}$. La tabla \ref{TablaCuantica} muestra la probabilidad sobre la línea hasta $t=5$. Toda posición que no es del extremo tiene amplitud igual a la suma de las amplitudes de sus primeros vecinos en el paso $t-1$. A propósito, los valores extremos sin interferencias tienen amplitud $p_{\text{extremos}}=1/2^t$. Para la posición junto a un extremo, la amplitud tiene relación con la amplitud de la posición $\ket{1}$ en $t=1$, como se puede comprobar directamente sobre la tabla trazando una diagonal de pendiente 1, que corresponde a algo así como un cono de luz (XXX): todos los valores incluidos dentro de las dos pendientes tienen influencia del valor inicial.\\

\begin{figure}[ht]
\centering
\includegraphics[width=1\textwidth]{Kap3/QWonlineNayak.png}
\caption{Simulación y función asintótica según la aproximación por el método de fase estacionaria, de una camianata con moneda de Hadamard y estado inicial $\ket{\downarrow,0}$, después de 100 pasos.}
\label{gr:Hadamard100Simulacion}
\end{figure}


\begin{table}[h]
    \centering
    \begin{tabular}{|c||c|c|c|c|c|c|c|c|c|c|c|c|}
        \hline
         &-5&-4&-3&-2&-1&0&1&2&3&4&5\\\hline\hline
        0&&&&&&1&&&&& \\ \hline
        1&&&&&$1/2$&&$1/2$&&&& \\\hline
        2&&&&$1/4$&&$1/2$&&$1/4$&&& \\ \hline
        3&&&$1/8$&&$5/8$&&$1/8$&&$1/8$&& \\ \hline
        4&&$1/16$&&$5/8$&&$1/8$&&$1/8$&&$1/16$& \\\hline
        5&$1/32$&&$17/32$&&$1/8$&&$1/8$&&$5/32$&&$1/32$ \\
    \hline
    \end{tabular}
    \medskip
    \caption{Las interferencias dan como resutado 
    probabilidades asimétricas alrededor de cero. Una mayor inteferencia negativa hace que el pico de la derecha sea menor que el de la iuerida, resultado de interferencias constructivas.}
    \label{TablaCuantica}
\end{table}
En las posiciones y tiempos $x+t$ impares la probabilidad es siempre nula, mientras que nunca lo es para $x+t$ par. La gráfica (\ref{gr:LineaHadamard100}) presenta la caminata en $t=100$ desde el estado inicial $\ket{\downarrow}\otimes\ket{0}$. 

\begin{figure}[ht]
\centering
\includegraphics[width=1\textwidth]{Kap3/comparisonQW.png}
\caption{La distribución gaussiana de la caminata clásica y la distribución bimodal de la caminata cuántica con moneda de Hadamard y estado inicial $\ket{\downarrow,0}$, después de 100 pasos.}
\label{gr:LineaHadamard100}
\end{figure}

\subsubsection{Grados de libertad de la moneda}\label{sec:Moneda}
Hemos analizado la caminata lineal con la moneda balanceada de Hadamard, que, sin embargo, genera una caminata asimétrica. En esta sección comprobamos que la evolución de la caminata depende tanto de la moneda como de la condición inicial. Después se muestra que ambas condiciones son equivalentes.
El estado inicial $\ket{\psi_{\text{sim}}}=\frac{1}{\sqrt{2}}(\ket{0}+i\ket{1})\ket{x=0}$ hace simétrica la caminata con la moneda de Hadamard. La evolución para cada estado de la superposición es independiente debido a la fase $i$. Sin embargo, debido a que la probabilidad es independiente de la fase, el resultado final es la suma de dos distribuciones para dos caminatas independientes, la primera con sesgo hacia la izquierda (fase inicial $0$) y la otra hacia derecha (fase inicial $\pi$), que juntas hacen la gráfica simétrica \ref{gr:Hadamard100Symmetric}. \\

\begin{figure}[ht]
\centering
\includegraphics[width=1\textwidth]{Kap3/QWonlinesymmetricNayak.png}
\caption{Simulación de una camianata con moneda de Hadamard y estado inicial $\ket{\psi_{\text{sim}}}=\frac{1}{\sqrt{2}}(\ket{0}+i\ket{1})\ket{x=0}$, después de 100 pasos.}
\label{gr:Hadamard100Symmetric}
\end{figure}

La misma gráfica se obtiene si usamos el estado inicial hacia izquierda y la siguiente moneda balanceada diferente a la de Hadamard \cite{kempe2003quantum},
\begin{equation*}
\hat{Y}\doteq
\begin{pmatrix}
1 & i\\
i & 1
\end{pmatrix}
\end{equation*}{}
$\hat{Y}\ket{\uparrow}=\dfrac{1}{\sqrt{2}}(\ket{\uparrow}+i\ket{\downarrow})$, $\hat{Y}\ket{\downarrow}=\dfrac{1}{\sqrt{2}}(i\ket{\uparrow}+\ket{\downarrow})$.\\

$[$Bach, 2004$]$ muestra que la moneda más general es de la forma
\begin{equation}
    C_2^{(\text{gen})}=
    \begin{pmatrix}
    \sqrt{\rho}&\sqrt{1-\rho}e^{e\theta}\\
    \sqrt{1-\rho}e^{i\phi} & -\sqrt{\rho}e^{i(\theta+\phi)}
    \end{pmatrix}{},
    \label{MonedaGeneral}
\end{equation}{}
donde $0\leq\theta,\phi\leq\pi$ son ángulos arbitrarios, $0\leq\rho\leq1$.
Una moneda balanceada como la de Fourier y una no balanceada como la de Grover.
\begin{equation}
    \hat{\mathcal{F}} C_2^{(\text{gen})}=
    \begin{pmatrix}
    \sqrt{\rho}e^{-ik}&\sqrt{1-\rho}e^{i(-k+\theta)}\\
    \sqrt{1-\rho}e^{i(k+\phi)} & -\sqrt{\rho}e^{i(k+\theta+\phi)}
    \end{pmatrix}{},
    \label{MonedaGeneral}
\end{equation}{}

\subsection{Caminos con borde}
Los caminos con borde son otro ejemplo de la diferencia entre las caminatas aleatorias clásicas y cuánticas. Recortemos la línea infinita y ubiquemos una pared absorbente que termina la caminata cuando la partícula choca con ella. Llamemos $0$ a la posición de esta pared y asumamos que el camino inicia en $1$. En el caso clásico, todos las caminatas posibles chocan con la pared, ¡ninguna la evita!. Aunque la línea es infinita y la probabilidad de recorrerla siempre hacia la derecha es diferente de cero, el análisis indica que la partícula siempre retorna y choca en $0$. La siguiente demostración que se hace por recurrencia: sea $p_{10}$ la probabilidad de llegar a $0$ por \textit{cualquier} camino; en particular, sabemos que tras una iteración, la probabilidad de ir a $0$ desde $1$ es $1/2$, y un valor igual el de ir hasta $2$. Ahora nos interesa $p_{20}$ que es la probabilidad de ir a $0$ desde $2$ a lo largo de \textit{cualquier} camino. Notemos que $p_{20}=p_{21}p_{10}$.

\begin{equation}
p_{10}=\frac{1}{2}+\frac{1}{2}p_{21}p_{10} ,
\label{CaminataBorde}
\end{equation}{}
La homogeneidad del espacio hace equivalentes todos los caminos que tienen distancia uno hacia la izquierda, en particular $p_{21}=p_{10}=p$. Teniendo en cuenta esto, la ecuación (\ref{CaminataBorde}) tiene solución para $p=1$. 
\begin{equation}
p=\frac{1}{2}+\frac{1}{2}p^2 ,
\end{equation}
Esto es equivalente a afirmar que la caminata clásica es \textit{recurrente}, esto es, que la partícula pasa por (choca) cada punto de la recta infinitas veces.\\
En el caso cuántico el estado de posición es una superposición de las posiciones posibles. Para saber si la partícula fue absorbida en el punto $b$, se necesita hacer una medición $M_b$ en $b$ (en la pared absorbente), tras cada paso $\hat{U}$,

\begin{equation*}
    \hat{M}_b\ket{\psi}=
    \left\{
    \begin{array}{cl}
        \ket{b} & \qquad p_b=|\bra{b}\ket{\psi}|^2 \\
        \dfrac{\ket{\psi}-\braket{b|\psi}\ket{b}}{\sqrt{1-|\braket{b|\psi}|^2}} &\qquad p_{B_\perp}=1-|\braket{b|\psi}|^2 
    \end{array}
    \right.
\end{equation*}{}
Tras hacer una medición se anulan las correlaciones, que no se pueden recuperar ya que éste es un proceso irreversible. $M_b$ proyecta sobre dos subespacios, el punto $b$ solo, o su complemento. Si se encuentra la partícula en $b$ se deja de medir. [Kempe, 2003b] presenta dos protocolos útiles que llevan a la misma conclusión: ¡la partícula escapa con probabilidad $2/\pi$!