%\newpage
%\setcounter{page}{1}
\begin{center}
\begin{figure}
\centering%
\epsfig{file=HojaTitulo/EscudoUN,scale=1}%
\end{figure}
\thispagestyle{empty} \vspace*{2.0cm} \textbf{\huge
Caminatas cuánticas}\\[6.0cm]
\Large\textbf{Roberto Germán Mesías Larrea}\\[6.0cm]
\small Universidad Nacional de Colombia\\
Facultad de Ciencias, Departamento de Física\\
Bogotá, Colombia\\
2019\\
\end{center}

\newpage{\pagestyle{empty}\cleardoublepage}

\newpage
\begin{center}
\thispagestyle{empty} \vspace*{0cm} \textbf{\huge
Caminatas cuánticas}\\[3.0cm]
\Large\textbf{Roberto Germán Mesías Larrea}\\[3.0cm]
\small Trabajo de grado presentado como requisito parcial para optar al
t\'{\i}tulo de:\\
\textbf{Físico}\\[2.5cm]
Director:\\
Ph.D., Carlos Leonardo Viviescas Ramirez\\[2.0cm]
L\'{\i}nea de Investigaci\'{o}n:\\
Computación Cuántica\\
Grupo de Investigaci\'{o}n:\\
Caos y Complejidad\\[2.5cm]
Universidad Nacional de Colombia\\
Facultad de Ciencias, Departamento de Física\\
Bogotá, Colombia\\
A\~{n}o\\
\end{center}

\newpage{\pagestyle{empty}\cleardoublepage}

\newpage
\thispagestyle{empty} \textbf{}\normalsize
\\\\\\%
\textbf{(Dedicatoria o un lema)}\\[4.0cm]

\begin{flushright}
\begin{minipage}{8cm}
    \noindent
        \small
        Su uso es opcional y cada autor podr\'{a} determinar la distribuci\'{o}n del texto en la p\'{a}gina, se sugiere esta presentaci\'{o}n. En ella el autor dedica su trabajo en forma especial a personas y/o entidades.\\[1.0cm]\\
        Por ejemplo:\\[1.0cm]
        A mis padres\\[1.0cm]\\
        o\\[1.0cm]
        La preocupaci\'{o}n por el hombre y su destino siempre debe ser el
        inter\'{e}s primordial de todo esfuerzo t\'{e}cnico. Nunca olvides esto
        entre tus diagramas y ecuaciones.\\\\
        Albert Einstein\\
\end{minipage}
\end{flushright}

\newpage{\pagestyle{empty}\cleardoublepage}

\newpage
\thispagestyle{empty} \textbf{}\normalsize
\\\\\\%
\textbf{\LARGE Agradecimientos}
\addcontentsline{toc}{chapter}{\numberline{}Agradecimientos}\\\\
Esta secci\'{o}n es opcional, en ella el autor agradece a las personas o instituciones que colaboraron en la realizaci\'{o}n de la tesis  o trabajo de investigaci\'{o}n. Si se incluye esta secci\'{o}n, deben aparecer los nombres completos, los cargos y su aporte al documento.\\

\newpage{\pagestyle{empty}\cleardoublepage}

\newpage
\textbf{\LARGE Resumen}
\addcontentsline{toc}{chapter}{\numberline{}Resumen}\\\\
El resumen es una presentaci\'{o}n abreviada y precisa (la NTC 1486 de 2008 recomienda revisar la norma ISO 214 de 1976). Se debe usar una extensi\'{o}n m\'{a}xima de 12 renglones. Se recomienda que este resumen sea anal\'{\i}tico, es decir, que sea completo, con informaci\'{o}n cuantitativa y cualitativa, generalmente incluyendo los siguientes aspectos: objetivos, dise\~{n}o, lugar y circunstancias, pacientes (u objetivo del estudio), intervenci\'{o}n, mediciones y principales resultados, y conclusiones. Al final del resumen se deben usar palabras claves tomadas del texto (m\'{\i}nimo 3 y m\'{a}ximo 7 palabras), las cuales permiten la recuperaci\'{o}n de la informaci\'{o}n.\\

\textbf{\small Palabras clave: (m\'{a}ximo 10 palabras, preferiblemente seleccionadas de las listas internacionales que permitan el indizado cruzado)}.\\

A continuaci\'{o}n se presentan algunos ejemplos de tesauros que se pueden consultar para asignar las palabras clave, seg\'{u}n el \'{a}rea tem\'{a}tica:\\

\textbf{Ciencia y tecnolog\'{\i}a}: 1) Astronomy Thesaurus Index. 2) Life Sciences Thesaurus, 3) Subject Vocabulary, Chemical Abstracts Service y 4) InterWATER: Tesauro de IRC - Centro Internacional de Agua Potable y Saneamiento.

\textbf{\LARGE Abstract}\\\\
Es el mismo resumen pero traducido al ingl\'{e}s. Se debe usar una extensi\'{o}n m\'{a}xima de 12 renglones. Al final del Abstract se deben traducir las anteriores palabras claves tomadas del texto (m\'{\i}nimo 3 y m\'{a}ximo 7 palabras), llamadas keywords. Es posible incluir el resumen en otro idioma diferente al espa\~{n}ol o al ingl\'{e}s, si se considera como importante dentro del tema tratado en la investigaci\'{o}n, por ejemplo: un trabajo dedicado a problemas ling\"{u}\'{\i}sticos del mandar\'{\i}n seguramente estar\'{\i}a mejor con un resumen en mandar\'{\i}n.\\[2.0cm]
\textbf{\small Keywords: palabras clave en ingl\'{e}s(m\'{a}ximo 10 palabras, preferiblemente seleccionadas de las listas internacionales que permitan el indizado cruzado)}\\