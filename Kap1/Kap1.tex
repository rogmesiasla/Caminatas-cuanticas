\chapter{Introducci\'{o}n}
Simular física cuántica en un computador convencional requiere recursos exponenciales. De allí la idea de construir un computador que manipule la información basado en los principios de la mecánica cuántica. Los computadores cuánticos, además de poder simular física cuántica, también podrán hacer operaciones notables con consecuencias importantes en diversas áreas. 
Es posible que la criptografía sea la aplicación más difundida, a la que se suman la simulación de átomos ultrafríos, el modelamiento de procesos químicos como posible método de desarrollo de nuevos fármacos. Sus características notables son consecuencia del paralelismo y la interferencia.\\

Un objetivo primordial es determinar cuándo los computadores pueden resolver problemas más rápidos que los computadores clásicos. En esta dirección,
los resultados más famosos son el algoritmo de Shor para factorizar números muy grandes cuya complejidad $\mathcal{O}(\log\log N)$, es una ventaja doble exponencial sobre los algoritmos clásicos (ya sean deterministas o probabilísticos), $\widetilde{\Theta}(\sqrt[4]{N})$, y el algoritmo de búsqueda de Grover que presenta una mejora cuadrática $\mathcal{O}(\sqrt{N})$ sobre el caso clásico $\mathcal{O}(N)$. De esta misma forma, las caminatas cuánticas pueden concretar mejoras exponenciales para diversos probelmas.\\

Las caminatas aleatorias modelan la traslación de un caminante sobre una distribución de vértices en la cual la dirección de cada paso se escoge como resultado de un evento aleatorio. 
Muchos modelos en la ciencia se basan en las caminatas aleatorias. En física la ecuación de Langevin o el movimiento browniano contienen distintos modelos de las caminatas aleatorias, en biología las descripciones de los movimientos de un individuo animal, en ciencias de la computación son una poderosa herramienta que sirve de base para la construcción de algoritmos.\\

Las caminatas cuánticas son procesos análogos a las caminatas aleatorias clásicas, lo que hace natural que presenten un alto poder práctico. El caso cuántico resulta interesante por varias razones: (1) sirven como modelo para procesos en diferentes áreas: por ejemplo la fotosíntesis, la difusión cuántica y el bombeo óptico [Kitakawa, 2005]; (2) mejoran el desempeño de los algoritmos clásicos; (3) son un modelo universal para computación cuántica [Childs, 2009, Lovett, 2010], equivalente, por ejemplo, al modelo de compuertas cuánticas de las máquinas de Turing; y (4) para su implementación no es necesario un computador cuántico.\\

Las caminatas clásicas y cuánticas presentan diferencias en varios aspectos y a varios niveles de sus análisis. Resalta, por su extrañeza, que la distribución de probabilidad del caso cuántico es bimodal a comparación de la distribución gaussiana. Además, y más importante, la varianza de la caminata cuántica es del orden del número de pasos $\mathcal{O}(t)$, mientras que el caso clásico es del orden de la raíz cuadrada del número de pasos $\mathcal{O}(\sqrt{t})$. Tras este resultado, es de esperar que la difusión sobre cualquier espacio que contenga la caminata sea mayor cuando es cuántica. Si a esto le sumamos cualidades como la superposición y la coherencia, las diferencias son definitivas: los algoritmos cuánticos presentan soluciones sin equivalente clásico a ciertos problemas, y mejoras en eficiencia inclusive exponenciales.\\

Sobre grafos las diferencias se evalúan en otros conceptos: algunos son el \textit{mixing time} que es el tiempo que tarda una caminata en converger hacia una distribución límite, y el \textit{hitting time} que es el tiempo en hallar un subconjunto marcado. En la búsqueda sobre el hipercubo (relacionado con el \textit{hitting time}), \cite{shenvi2003quantum} mostró que las caminatas cuánticas son capaces de mejorar cuadráticamente el desempeño de \textit{cualquier} algoritmo clásico. Por su parte, \cite{childs2003exponential} encontró un algoritmo que propaga al caminante, sobre un tipo de grafo de árboles ligados, exponencialmente más rápido que \textit{cualquier} algoritmo clásico. En estos ejemplos la complejidad es mejor no sólo a comparación de las contrapartes clásicas basadas en caminatas aleatorias, pero de cualquier tipo de algoritmo.\\

Con base en caminatas sobre grafos adecuados, las caminatas cuánticas son capaces de resolver problemas de la computación como: búsqueda espacial, problema de colisión sobre grafos, problema de distinción de elementos, \textit{NAND tree evaluation}, \textit{single-source shortest path},   \textit{triangle finding problem}, etc. \cite{shao}. Ambainis et. al \cite{ambainis2007quantum} obtuvieron un algoritmo óptimo para resolver el problema de la distinción de elementos que usa caminatas cuánticas. Szegedy \cite{szegedy2004quantum} propuso un marco general para las caminatas sobre grafos valioso para la búsqueda espacial. Estos dos algoritmos fueron un modelo para la construcción de otros. En cualquier cadena de Markov simétrica y ergódica, el marco de Szegedy logra una mejora cuadrática sobre cualquier algoritmo clásico. Otra marco de búsqueda común es el desarrollado por Magníez et. al \cite{magniez2011search}, llamado MNRS, con el que se descubrieron algoritmos veloces como el de \textit{triangle finding problema} que busca reconocer una estructura triangular en alguna parte del grafo, \textit{group commutative test}, etc.\\

En el trabajo exploramos las caminatas sobre la línea y sobre grafos finitos e infinitos, y resaltamos sus similitudes y diferencias, con énfasis en aquello que destaca a la caminata cuántica en la construcción de los algoritmos.
Está dividido en cuatro partes. En la primera presentamos los modelos las características fundamentales de las caminatas clásicas y las cadenas de Markov. En la parte dos estudiamos la caminata cuántica sobre la línea y definimos los elementos básicos para el análisis y caracterizamos el operador de evolución temporal.
Posteriormente, en la parte tres, las caminatas sobre redes finitas y su dinámica, y presentamos la solución particular para ciertas gráficas. En la cuarta parte, presentamos algunos algoritmos con atención especial en algunos marcos técnicos para los problemas de búsqueda espacial.\\

La revisión de algunos conceptos básicos de la mecánica cuántica: el carácter determinista de la evolución unitaria, la aleatoriedad asociada a la medida, la superposición natural en los espacios de Hilbert, el entrelazamiento entre grados de libertad de un mismo sistema, y los niveles de decoherencia como puente entre lo clásico y lo cuántico, que puede ser un soporte técnico para la optimización de las caminatas. La implementación de las caminatas en algoritmos cuánticos presenta ideas básicas de computación cuántica. En el apéndice se presentan el algoritmo de Grover, óptimo para búsqueda en un conjunto desordenado, algo del álgebra para computación cuántica y circuitos cuánticos.\\
